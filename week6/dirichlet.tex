\documentclass[twoside]{article}
\usepackage{amsmath,amssymb,amsthm,graphicx}
\usepackage{epsfig}
\usepackage[authoryear]{natbib}

\usepackage{geometry}
\usepackage{setspace}

\geometry{twoside,
          letterpaper, % i.e, paperwidth=210mm and paperheight=297mm,
          top=25mm,
          bottom=45mm,
          left=25mm,
          right=25mm,
}

\setlength{\parindent}{0pt}
\setlength{\parskip}{0.5cm}
% Local Macros Put your favorite macros here that don't appear in
% stat-macros.tex.  We can eventually incorporate them into
% stat-macros.tex if they're of general use.

\begin{document}

\textbf{Reflection - Dirichlet reflection - Frigyik et al.}\\
\textbf{Nicholas Hoernle \hfill \today}

\begin{enumerate}
  \item The $GEM$ distribution with concentration parameter $\alpha$, is closely related to the Dirichlet stick-breaking process. For a $K$ dimensional Dirichlet distribution, with parameters $a_{1:K}$, then a draw ($\rho$) from the Dirichlet distribution also can be written as a stick breaking process with $K-1$ breaks (and $K$ pieces):\\ $\rho_1 \sim Beta(\alpha_1, \sum\limits_{k=1}^K(a_k-a_1)), \frac{\rho_2}{1-\rho_1} \sim Beta(\alpha_2, \sum\limits_{k=2}^K(a_k-a_2)), \hdots, \rho_K = 1-\sum\limits_{k=1}^{K-1}\rho_k$. If we did not place a limit on $K$ and rather let $K \rightarrow \infty$, and if all $a_k = \alpha \forall K$ then we have described the $GEM$ distribution. So the $GEM$ distribution is like an infinite dimensional Dirichlet distribution.
  \item Two elements drawn from the $GEM$ $\rho_i$ and $\rho_j$ for $i < j$ are not independent. Consider the stick breaking convention for the $GEM(\alpha)$ distribution where $\rho_k = \left[ \prod\limits_{j=1}^{k-1}(1-V_j) \right] V_k = \left\{ \left[ \prod\limits_{j=1}^{k-2}(1-V_j) \right] - \rho_{k-1} \right\} V_k$ (where $V_k$ is the proportion that is broken off at each step). Here we can see that $\rho_k$ depends on the value of $\rho_{k-1}$ (and previous values of $\rho_i$) and thus for two samples $\rho_i$ and $\rho_j$ for $i < j$, $\rho_j$ is dependent on $\rho_i$ ($P(\rho_{j} \mid \rho_{i}) \neq P(\rho_{j})$).
  \item As $N \rightarrow \infty$, the number of unique values $\{ z_n \}$ also tends to $\infty$. This is exactly the paradigm that we desire as we'd like the number of assignments to grow appropriately with more observations of data (e.g. as the number of documents in a corpus tends to $\infty$, we might need new and unseen topics to help describe the data).
\end{enumerate}
\end{document}
