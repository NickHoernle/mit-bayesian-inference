\documentclass[twoside]{article}
\usepackage{amsmath,amssymb,amsthm,graphicx}
\usepackage{epsfig}
\usepackage[authoryear]{natbib}

\usepackage{geometry}
\usepackage{setspace}

\geometry{twoside,
          letterpaper, % i.e, paperwidth=210mm and paperheight=297mm,
          top=25mm,
          bottom=40mm,
          left=25mm,
          right=25mm,
}

\setlength{\parindent}{0pt}
\setlength{\parskip}{0.5cm plus4mm minus3mm}

\begin{document}

\textbf{Reflection - Project Progress Report}\\
\textbf{Nicholas Hoernle \hfill \today}

\textbf{Work done to-date}

I have explored the Dirichlet process~(DP), hierarchical Dirichlet process~(HDP) and hierarchical Dirichlet process for hidden Markov models~(HDP-HMM) literature. To assist my understanding of the DP, I wrote a generative model of a mixture of 4 Gaussians and I have written a basic Stan model for inference over the parameters (with a stick breaking prior). The implementation marginalizes over the cluster assignments ($z_i$) for compatability with the NUTS sampler and uses a posterior predictive step to make the cluster assignments to data. The Stan implementation is exceedingly slow but the implementation was useful for my understanding. I used a truncated DP here for inference.

I have also implemented a blocked Gibbs sampler for the sticky hierarchical Dirichlet process for Hidden Markov Models (HDP-HMM) as specified in Algorithm 2 of \cite{fox2007developing} and discussed further in \cite{fox2008hdp}. I have run a test on a small generated time series where sequential data are generated from three Gaussian modes. The sticky HDP-HMM recovers the cluster assignments and regime means as is presented in \cite{fox2007developing}.

% The HDP can be applied to a HMM with unknown state space cardinality. If there are countably infinitely many HMM state values, for each state, there is a countably infinite transition density over the next HMM state. Note that the transition densities are state specific and thus the hierarchical extension to the DP is necessary.

\textbf{Plan for rest of project}

The rest of the project aims to implement the Gibbs samplers for the HDP-SLDS and HDP-AR-HMM models from \cite{fox2011bayesian}. I will begin by implementing and reproducing the tests on generated data for the HDP-AR-HMM model. If time permits, I will expand this implementation to that for the HDP-SLDS.
\begin{enumerate}
  \item Implement the forward and backward recursions for the SSM (the ones for the HMM have already been done).
  \item Write the Gibbs sampler that is presented for \cite{fox2011bayesian} for the HDP-AR-HMM model.
  \item Expand upon (2) to develop a Gibbs sampler for the HDP-SLDS model (if time permits).
  \item Reproduce the tests from \cite{fox2011bayesian} on their generated data.
\end{enumerate}

\bibliographystyle{abbrv}
\bibliography{references}

\end{document}
